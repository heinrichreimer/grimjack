\begin{abstract}
%  We, Team Grimjack, present our approach for answering comparative questions, submitted in five runs to the Touché Lab on Argument Retrieval.
%  Our approach follows two objectives: ranking argumentative and high-quality documents first, and exposing arguments of different stances towards the compared objects fairly in high ranking positions.
%  We therefore propose a multi-stage retrieval pipeline with query reformulation and combination, baseline retrieval, quality and stance tagging, and different task-specific re-ranking steps.
%  First, we re-rank axiomatically based on argumentative retrieval axioms.
%  Second, we re-rank to ensure fair exposure across argument stances.
%  In all retrieval steps, we use the T0 language model to evaluate whether zero-shot language models can successfully answer comparative questions.
In this paper, we present the Team's Grimjack retrieval approaches for the Touch{\'e} shared task on Argument Retrieval for Comparative Questions. In total, we submitted five runs that pursue the two main objectives: favoring argumentative and high argument quality documents in the final ranking and addressing a retrieval ``fairness'' by ensuring an even ratio of pro and con arguments at top ranks.
%proposed alternative that might be enough for now. Once we have official evaluation results, we can add a sentence or two on that.
\end{abstract}

\begin{keywords}
  Axiomatic Re-ranking \sep
  Query Reformulation \sep
  T0 \sep
  Argument Quality \sep
  Argument Stance \sep
  CEUR-WS
\end{keywords}
