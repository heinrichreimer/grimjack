\section{Conclusion}

% We propose a flexible approach to tackle the task of answering comparative questions and detecting the stance of the returned passages with respect to the comparative objects.
To retrieve relevant argumentative passages, we combine query reformulation and expansion techniques with axiomatic re-ranking exploiting argumentative structure, quality, and stance.
Using the IBM Debater API and the T0++~language model, we showcase two state-of-the-art approaches for argument quality tagging.
We extend previous query expansion approaches from the Touché shared tasks by incorporating the contextual information provided in topic descriptions and narratives.
To attain fair exposure in the final ranking, we balance the pro and con arguments on top-10 ranks.

Our experiments indicate that query expansion and reformulation can provide better search results for some topics, but can also decrease effectiveness for others.
A subsequent axiomatic re-ranking step can improve the system's precision but we do not include a systematic evaluation of the re-ranker. \todo{Evaluate axiomatic re-ranking.}

Because our approach for stance classification depends on an API for single-target stance classification, aggregating the objects' single-target stances to form a multi-target stance is not straight-forward.
We did not find a natural way to distinguish neutral arguments from arguments with no stance.
Arguably, fine-tuning a multi-class neural classifier like \Bert on the stance dataset provided by \citeauthor{BondarenkoFKSGBPBSWPH2022} could improve classification performance by directly predicting the multi-target stance.

Our runs featuring text generation with the T0++ languge model motivate the controversial question whether we could effectively retrieve arguments while using large language models in every step of the retrieval pipeline.
Given the recent doubts about the explainability of such models, we look forward to evaluate this question in more detail with relevance judgements for the Touché 2022 shared task.
