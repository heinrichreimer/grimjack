\section{Related Work}

Personal decision making often starts with formulating comparative questions like ``Should I major in philosophy or psychology?''~\cite{BondarenkoFBGAPBSWPH2020,BondarenkoGFBAPBSWPH2021,BondarenkoFKSGBPBSWPH2022}. Short direct answers (potentially biased)~\cite{PotthastHS2020} to such questions might be insufficient; instead, such questions require diverse opinions to provide a sufficient, balanced, and argumentative overview~\cite{BondarenkoFBGAPBSWPH2020}.
The Touch{\'e} shared task on Argument Retrieval for Comparative Questions was proposed to evaluate retrieval approaches on a large corpus with respect to relevance and rhetorical quality of potential answers to comparative questions that also may represent different standpoints~\cite{BondarenkoFKSGBPBSWPH2022,BondarenkoADHBH2022}.

The most effective approaches at previous Touch{\'e} editions~\cite{BondarenkoFBGAPBSWPH2020,BondarenkoGFBAPBSWPH2021} successfully used query expansion with synonyms and antonyms~\cite{AbyeST2020}, identified premises and claims in retrieved documents~\cite{Huck2020, ShirshakovaW2021}, estimated argument quality~\cite{AbyeST2020}, and re-ranked initially retrieved documents based on argument quality and document ``comparativeness'', e.g., a ratio of comparative adjectives~\cite{ChekalinaP2021}. Inspired by the participant approaches from the previous Touch{\'e} editions, we also include the components of argument mining and argument quality estimation in our retrieval pipeline, however, using different methods.
We rely on a large language model~T0 trained in multitask setting that showed to achieve state-of-the-art results for various Natural Language Processing tasks in zero-shot settings~\cite{SanhWRBSACSLRDBXTSSKCNDCJWMSYPBWNRSSFFTBGBWR2021}. The largest pretrained T0 variant, T0++, was trained on 62~datasets with 12~task-specific prompts covering such tasks as question answering, sentiment analysis, summarization, etc. By using T0++, we aim for answering a question whether the abilities of large language models are sufficient for the new task of argument retrieval.

Our second idea of axiomatic re-ranking comes from axiomatic thinking in information retrieval, where axioms formally describe constraints that good retrieval model should fulfil, e.g., documents with more query term occurrences should be ranked higher~\citet{FangTZ2004}. It has already been shown that combining multiple axioms for re-ranking results of arbitrary retrieval models can improve final overall retrieval effectiveness~\cite{HagenVGS2016}. Complementing existing retrieval axioms, \citet{BondarenkoHVSPB2018} introduced argumentativeness axioms based on claims and premises in documents identified with TARGER~\cite{ChernodubOHBHBP2019}.
