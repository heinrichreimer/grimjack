\section{Conclusion}
    We have seen an approach to tackle the task of answering comparative questions and detecting the stance of the returned passages regarding the comparative objects. Our approach combines query reformulation/expansion techniques with axiomatic reranking and shows two princuple ways of determing the argument quality and stance. Our work expands previous work from earlier years of the shared task by using different techniques to expand the orriginal queries and also incorporates the information from the description and narrative field. We also use axiomatic reranking in contrast to reranking based soley on weighting scores.\par
    We have seen in chapter \ref{evaluation} that query expansion/reformulation and the additional information from the narrative and description field can provide a better search result. But it is also possible that the reult gets slightly worse in comparsion to not use query expansion/reformulation all. This is due to the fact that with these techniques we will alter the search terms used and some synonyms could lead to passages which do not comply with the user's information need. The axiomatic reranking proves to be robust and reliable but sometimes it can not compensate the erros from the query expansion/reformulation and passages which were relevant will be further down in the ranking.\par 
    Difficulties arise for our stance classification since we soley rely on external APIs which only provide single target stance. Firstly, computing the multi target stance from single target stance proved to be challenging. Second, our approach can not decide between no argument present and neutral stance which is a big caveat. The performance of our classification could be improved by using BERT models or other machine learning based approaches which are able to predict a multi target stance. A interessting question arised during our research: Is it possible to only use the language model T0 to develop a information retrieval pipeline?\par
    There are many different challenging problems to solve to create an information retrieval pipeline to answer comparative questions. Our approach proposed serveral possibilities how to solve them each with their own limitations and caveats.    