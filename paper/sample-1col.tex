%% The first command in your LaTeX source must be the \documentclass command.
%%
%% Options:
%% twocolumn : Two column layout.
%% hf: enable header and footer.
\documentclass[
% twocolumn,
% hf,
]{ceurart}

%%
%% One can fix some overfulls
% \sloppy

%%
%% Minted listings support 
%% Need pygment <http://pygments.org/> <http://pypi.python.org/pypi/Pygments>
% \usepackage{minted}
%% auto break lines
% \setminted{breaklines=true}
%%
%% end of the preamble, start of the body of the document source.
\begin{document}

%%
%% Rights management information.
%% CC-BY is default license.
\copyrightyear{2022}
\copyrightclause{Copyright for this paper by its authors.
	Use permitted under Creative Commons License Attribution 4.0
	International (CC BY 4.0).}

%%
%% This command is for the conference information
\conference{CLEF 2021 -- Conference and Labs of the Evaluation Forum, 
	September 5--8, 2022, Bologna, Italy}

%%
%% The "title" command
%% Do not remove the subtitle Notebook for the Touch{\'e} Lab on Argument Retrieval at CLEF 2021
\title{Answer Comparative Questions with Query Reformulation and Axiomatic Reranking}
\title[mode=sub]{Notebook for the Touch{\'e} Lab on Argument Retrieval at CLEF 2022}

%%
%% The "author" command and its associated commands are used to define
%% the authors and their affiliations.
\author[1]{Jan Heinrich Reimer}[%
email=jan.reimer@student.uni-halle.de,
url=https://github.com/heinrichreimer,
]
\address[1]{Martin-Luther-University Halle-Wittenberg,
Universitätsplatz 10, Halle, 06108, Germany}

\author[1]{Johannes Huck}[%
email=johannes.huck@student.uni-halle.de,
url=https://github.com/johannes-huck,
]

\begin{abstract}
%  We, Team Grimjack, present our approach for answering comparative questions, submitted in five runs to the Touché Lab on Argument Retrieval.
%  Our approach follows two objectives: ranking argumentative and high-quality documents first, and exposing arguments of different stances towards the compared objects fairly in high ranking positions.
%  We therefore propose a multi-stage retrieval pipeline with query reformulation and combination, baseline retrieval, quality and stance tagging, and different task-specific re-ranking steps.
%  First, we re-rank axiomatically based on argumentative retrieval axioms.
%  Second, we re-rank to ensure fair exposure across argument stances.
%  In all retrieval steps, we use the T0 language model to evaluate whether zero-shot language models can successfully answer comparative questions.
In this paper, we present the Team's Grimjack retrieval approaches for the Touch{\'e} shared task on Argument Retrieval for Comparative Questions. In total, we submitted five runs that pursue the two main objectives: favoring argumentative and high argument quality documents in the final ranking and addressing a retrieval ``fairness'' by ensuring an even ratio of pro and con arguments at top ranks.
%proposed alternative that might be enough for now. Once we have official evaluation results, we can add a sentence or two on that.
\end{abstract}

\begin{keywords}
  Axiomatic Re-ranking \sep
  Query Reformulation \sep
  T0 \sep
  Argument Quality \sep
  Argument Stance \sep
  CEUR-WS
\end{keywords}


%%
%% This command processes the author and affiliation and title
%% information and builds the first part of the formatted document.
\maketitle

\section{Introduction}\label{intro}

Argument retrieval is a specific task that not only considers topical relevance of retrieved documents to given queries (usually of controversial, argumentative or opinion nature) but also accounts for argument specific features like argument quality and stance~\cite{BondarenkoFBGAPBSWPH2020, BondarenkoGFBAPBSWPH2021}.  
Furthermore, it has been shown that current search engines might return biased results~\cite{ShahB2022} and argument retrieval systems return ``unfairelly'' distributed pro / con arguments~\cite{CherumanalSSC2021}.
We especially emphasize the importance of retrieving diverse results for comparative questions (e.g., ``Train or plane? Which is the better choice?'') that provide different point of views to mitigate biasing users' decisions towards one or the other comparison option.

Our Team Grimjack participated in the Touch{\'e} shared task on Argument Retrieval for Comparative Questions which goals are: \Ni To retrieve relevant and high quality argumentative passages from a collection of 868\,655~text passages to a set of 50~search topics and \Nii to classify the stance of the retrieved passages towards the comparison objects in search topics~\cite{BondarenkoFKSGBPBSWPH2022}.
As part of our participation in the task, we have developed a flexible retrieval pipeline in Python based on Pyserini~\cite{LinMLYPN2021} as an easily configurable command line application.
In the first step, our approach uses query (comparative questions from topics' titles) reformulation and expansion by important terms from topics' descriptions and narratives. Then the top~100 initially retrieved passages using query likelihood with Dirichlet smoothing~\cite{ZhaiL2001} are axiomatically re-ranked based on the number and position of premises, claims, and comparative objects (identified with TARGER~\cite{ChernodubOHBHBP2019}) and argument quality predictions by the IBM Debater API~\cite{ToledoGCFVLJAS2019} and T0++~\cite{SanhWRBSACSLRDBXTSSKCNDCJWMSYPBWNRSSFFTBGBWR2021}.
Finally, the pro and con argumentative passages towards the comparative objects are balanced in the final ranking by alternating documents of different stance (cf.\ Section~\ref{approach} for more details on the approach and submitted runs).

Additionally, we manually labeled the \todo{relevance} of \todo{XX} documents for three topics returned by our system (before the official results are made available by the organizers). Our manual assessment shows the potential of expanding queries with synonyms and contextual information for improving the effectiveness of our argument retrieval approach~(cf.\ Section~\ref{evaluation}).

%% The last statement could fit in the CR version depending on the final evaluation: good or bad
%Different configurations for our submitted runs should pose examples to discuss current doubts about the usefulness of large zero-shot language models like T0++~\cite{SanhWRBSACSLRDBXTSSKCNDCJWMSYPBWNRSSFFTBGBWR2021} in the field of argumentative information retrieval~\cite{ShahB2022}.

\section{Related Work}

Personal decision making often starts with formulating comparative questions like ``Should I major in philosophy or psychology?''~\cite{BondarenkoFBGAPBSWPH2020,BondarenkoGFBAPBSWPH2021,BondarenkoFKSGBPBSWPH2022}. Short direct answers (potentially biased)~\cite{PotthastHS2020} to such questions might be insufficient; instead, such questions require diverse opinions to provide a sufficient, balanced, and argumentative overview~\cite{BondarenkoFBGAPBSWPH2020}.
The Touch{\'e} shared task on Argument Retrieval for Comparative Questions was proposed to evaluate retrieval approaches on a large corpus with respect to relevance and rhetorical quality of potential answers to comparative questions that also may represent different standpoints~\cite{BondarenkoFKSGBPBSWPH2022,BondarenkoADHBH2022}.

The most effective approaches at previous Touch{\'e} editions~\cite{BondarenkoFBGAPBSWPH2020,BondarenkoGFBAPBSWPH2021} successfully used query expansion with synonyms and antonyms~\cite{AbyeST2020}, identified premises and claims in retrieved documents~\cite{Huck2020, ShirshakovaW2021}, estimated argument quality~\cite{AbyeST2020}, and re-ranked initially retrieved documents based on argument quality and document ``comparativeness'', e.g., a ratio of comparative adjectives~\cite{ChekalinaP2021}. Inspired by the participant approaches from the previous Touch{\'e} editions, we also include the components of argument mining and argument quality estimation in our retrieval pipeline, however, using different methods.
We rely on a large language model~T0 trained in multitask setting that showed to achieve state-of-the-art results for various Natural Language Processing tasks in zero-shot settings~\cite{SanhWRBSACSLRDBXTSSKCNDCJWMSYPBWNRSSFFTBGBWR2021}. The largest pretrained T0 variant, T0++, was trained on 62~datasets with 12~task-specific prompts covering such tasks as question answering, sentiment analysis, summarization, etc. By using T0++, we aim for answering a question whether the abilities of large language models are sufficient for the new task of argument retrieval.

Our second idea of axiomatic re-ranking comes from axiomatic thinking in information retrieval, where axioms formally describe constraints that good retrieval model should fulfil, e.g., documents with more query term occurrences should be ranked higher~\citet{FangTZ2004}. It has already been shown that combining multiple axioms for re-ranking results of arbitrary retrieval models can improve final overall retrieval effectiveness~\cite{HagenVGS2016}. Complementing existing retrieval axioms, \citet{BondarenkoHVSPB2018} introduced argumentativeness axioms based on claims and premises in documents identified with TARGER~\cite{ChernodubOHBHBP2019}.

\section{Approach}\label{approach}

\begin{figure}
	\centering
    \begin{tikzpicture}[
        xscale=3.25,
        yscale=-1.75,
        block/.style={
            rectangle,
            draw,
            fill=white,
            text centered,
            rounded corners=1mm,
            text width=5.8em,
            minimum height=2em
        },
        line/.style={
            draw,
            -Latex,
            rounded corners
        },
        ]

        \node (documents) at (0,0) {Docs.};
        \node [block] (index) at (1,0) {Indexing};
        \node (topics) at (0,1) {Topics};
        \node [block,dashed,draw opacity=0.5,xshift=1mm,yshift=1mm] at (1,1) {Query Expansion / \\ Reformulation};
        \node [block,dashed] (query-expansion-reformulation) at (1,1) {Query Expansion / \\ Reformulation};
        \node [block] (query-combining) at (2,1) {Query Combination};
        \node [block] (retrieval) at (3,0.5) {Retrieval};
        \node [block] (quality-tagging) at (1,2.5) {Quality Tagging};
        \node [block] (stance-tagging) at (2,2.5) {Stance Tagging};
        \node [block,dashed] (axiomatic-reranking) at (1,4) {Axiomatic Re-ranking};
        \node [block,dashed] (fairness-reranking) at (2,4) {Fairness Re-ranking};
        \node (ranking) at (3,4) {Ranking};
ranking
        \path [line] (documents) -- (index);
        \path [line] (topics) -- (query-expansion-reformulation);
        \path [line] (query-expansion-reformulation) -- (query-combining);
        \path [line] (index) -| (retrieval);
        \path [line] (query-combining) -| (retrieval);
        \path [line] (retrieval) -| (3.5,1.5) |- (2,1.75) -| (0.5,2) |- (quality-tagging);
        \path [line] (quality-tagging) -- (stance-tagging);
        \path [line] (stance-tagging) -| (2.5,3) |- (2,3.25) -| (0.5,3.5) |- (axiomatic-reranking);
        \path [line] (axiomatic-reranking) -- (fairness-reranking);
        \path [line] (fairness-reranking) -- (ranking);
    \end{tikzpicture}\\[0.75em]
	\caption{Architecture overview for the retrieval pipeline used to produce our runs. Dashed boxes indicate optional steps, that are not used in all runs.}
    \label{figure-pipeline}
\end{figure}


We design the architecture of our argumentative retrieval system as a pipeline of multiple steps that subsequently (re-)rank, annotate, or modify the documents or queries given as inputs. This pipeline is shown in Figure~\ref{figure-pipeline}.
We identify four core steps as most important to our approach:
\Ni query expansion, reformulation, and combination,
\Nii first-stage retrieval,
\Niii argument quality and stance tagging,
and \Niv axiomatic reranking and fairness reranking.

Additionally, we add an evaluation component that is not shown in Figure~\ref{figure-pipeline} because it is not needed to retrieve results from our system.
With this evaluation component we can evaluate our system on topics of previous editions~(i.e.,~2021 and~2020) of the Touché Lab on Argument Retrieval~(c.f.~Section~\ref{transfer-relevance-judgements}).

\subsection{Query Expansion, Reformulation, and Combination}
\label{reformulation}

In order to increase recall of our first-stage ranker and to include results for very similar yet differently named objects, we first expand and reformulate the original search query.
Our approaches use two different strategies: \Ni we replace the comparative objects with their synonyms and \Nii we generate additional, new queries using the additional description and narrative information provided by the shared task organizers~\cite{BondarenkoFKSGBPBSWPH2022}.
Expanding the original query with synonyms of comparative objects is motivated by the fact that documents often contain more specific comparisons~(e.g., Ubuntu vs Windows) instead of more broad comparisons~(e.g., Linux vs Windows).
Yet, specific examples of a more general class of objects are useful to answer comparative questions about their object class.
We might therefore find relevant documents that would otherwise not match any original query term.
It is however important to note that increasing recall can result in an decrease in precision which is undesirable in the precision-oriented setting of the shared task.
However, we later apply re-ranking steps that improve precision by moving irrelevent documents further down the ranking~(c.f.~Section~\ref{reranking}).

\paragraph{Query Expansion with Synonyms}

We use two different strategies to find synonyms: \Ni word embeddings and \Nii a zero-shot language model.
In our first strategy, we use fastText word embeddings~\cite{BojanowskiGJM2017} to find words with the highest cosine similarity to the given comparative objects in the embedding space.
We manually examine synonyms from fastText embeddings using different domains~(i.e., Wikipedia and Twitter) and find that fastText embeddings trained on the Twitter corpus result in the best synonyms.

Our second strategy to obtain synonyms is based on the T0++~zero-shot language model~\cite{SanhWRBSACSLRDBXTSSKCNDCJWMSYPBWNRSSFFTBGBWR2021}.
We ask the model to generate an answer to the following task: \query{What are synonyms of the word~<token>?} where \query{<token>}~is one of the two comparative objects.
From the text returned by the language model, we then parse synonyms by splitting at commas and remove duplicate synonyms.
With the synonyms returned by either strategy, we replace the comparative objects to form new queries. All alternative queries and the original query are then combined.

\paragraph{Query Reformulation with Topic Context}
\begin{table}
    \caption{Original queries provided by Touch{\'e} and generated queries by T0++~\cite{SanhWRBSACSLRDBXTSSKCNDCJWMSYPBWNRSSFFTBGBWR2021} using the topic's description~(D) or narrative~(N).}
    \label{table-generated-queries}
    \begin{tabularx}{\linewidth}{c >{\hsize=.7\hsize\linewidth=\hsize}X c >{\hsize=1.3\hsize\linewidth=\hsize}X}
        \toprule
        \textbf{Topic} & \textbf{Original query} & \textbf{Field} & \textbf{Generated query} \\
        \midrule
        \multirow{2}{*}{12} & \multirow{2}{\linewidth}{Train or plane? Which is the better choice?} & D & Travel \\
        & & N & What are the benefits of trains over planes for intercontinental travel? \\
        \addlinespace
        \multirow{2}{*}{53} & \multirow{2}{\linewidth}{Should I buy steel or ceramic knives?} & D & Why should I choose ceramic knives over steel knives? \\
        & & N & What are the pros and cons of ceramic knives? \\
        \addlinespace
        \multirow{2}{*}{88} & \multirow{2}{\linewidth}{Should I major in philosophy or psychology?} & D & What is the difference between philosophy and psychology? \\
        & & N & What are the benefits of a major in English or history? \\
        \addlinespace
        \multirow{2}{*}{95} & \multirow{2}{\linewidth}{Which is more environmentally friendly, a hybrid or a diesel?} & D & What are the most environmentally friendly cars? \\
        & & N & What is more environmentally friendly, a diesel or a hybrid car? \\
        \bottomrule
    \end{tabularx}
\end{table}


We complement the queries expanded by replacing synonyms with newly generated queries that incorporate the contextual information provided in description narrative fields from the shared task's topics.
The description contains important details about the actual information need and the narrative clearly defines which passages are relevant for the query.
We use this valuable information about which passages to retrieve by generating new queries with the T0++ language model~\cite{SanhWRBSACSLRDBXTSSKCNDCJWMSYPBWNRSSFFTBGBWR2021} and providing it with a topic's description or narrative.

We challenge the T0++ model with the following task: \query{<text>.~Extract a natural search query from this description.} where \query{<text>}~is either the topic's narrative or description.
The string returned by the language model is then used as is as the new query for the topic and combined with the previous queries.
In Table~\ref{table-generated-queries}, we show examples of generated queries.
Albeit some of the generated queries~(e.g., topic~53) are just reformulations of the original query, the T0++ language model gernerates some interesting new queries for other topics~(e.g., topic~12).

\paragraph{Disjunctive Query Combination}
After the query expansion and query reformulation steps, we need to combine all computed queries and the original query.
We decide to combine all queries in a single query in a logical disjunction, that is by using Pyserini's OR operator.
Two reasons influence this decision:
Firstly, retrieving results for just one query is conceptually easier as we don't neet to interleave multiple result sets after the retrieval step.
Interleaving is not trivial and it is often challenging to find an interleaving strategy without many caveats.
Secondly, the logical disjunction increases the system's recall and decrease the chance of empty result sets in the cse that a term is not present in the corpus.
Although, the query reformulation, expansion and combination steps are optional, meaning that we use these steps only in some runs. In most of our submitted runs, we just use the original query, because the increase in recall might result in a decreasequery in precision that is hard to offset in subsequent re-ranking steps.

\subsection{Passages Retrieval}\label{retrieval}

To retrieve passages from the set of passages extracted from ClueWeb~12 by \citet{BondarenkoFKSGBPBSWPH2022}, we first build an inverted index using the Pyserini framework~\cite{LinMLYPN2021}.
Pyserini allows for experimenting with multiple steps of a retrieval system including indexing and simple retrieval models like Okapi~BM25 or the query likelihood model.
In the index, we store index term positions, passage vectors, and raw passage contents.
Index terms are stemmed using the Porter stemmer~\cite{Porter1980} and stop words are removed as per the default Pyserini stopword list~\cite{LinMLYPN2021}.
We then retrieve passages for the previously combined query~(c.f. Section~\ref{reformulation}) using the query likelihood model with Dirichlet smoothing~(\( \mu = 1000 \)) in Pyserini.
From this first-stage ranker, we retrieve 100~candidate passages for each query.

\subsection{Argument Quality and Stance Tagging}

After retrieving candidate passages, we tag the argumentative structure, argumentative quality and argument stance.
Argument structure, quality, and stance are required for later steps in our retrieval pipeline to re-rank the passages~(c.f. Section~\ref{reranking}).
Also, the task organizers ask the participants to optionally return a stance for each retrieved document as a sub-task in the Touché Lab on Argument Retrieval.
We tag each passage's argumentative structure with the TARGER API\footnote{\url{https://demo.webis.de/targer/}} using the \textttsmall{targer-api} Python package\footnote{\url{https://github.com/webis-de/targer-api}}.
In order to tag each passage's quality and stance we first split each retrieved passage into sentences using the NLTK library~\cite{BirdLK2009}.
Then each sentence is treated as one potential argument and tagged with its argumentative quality and stance.
To find the quality score and stance for the whole passage, we average the quality or stance scores respectively of all sentences in the passage.

\paragraph{Argument Quality Tagging}
\begin{table}
    \caption{Argument quaity label mapping for textual labels returned by the T0++~language model~\cite{SanhWRBSACSLRDBXTSSKCNDCJWMSYPBWNRSSFFTBGBWR2021}.}
    \label{table-quality-mapping}
    \begin{tabular}{lc}
        \toprule
        \textbf{Text Label} & \textbf{Value} \\
        \midrule
        \query{very good} & 1.00 \\
        \query{good} & 0.75 \\
        \query{bad} & 0.25 \\
        \query{very bad} & 0.00 \\
        other & 0.50 \\
        \bottomrule
    \end{tabular}
\end{table}


We implement two different methods for quality tagging:
Our first method is based on the IBM Debater API~\footnote{\url{https://early-access-program.debater.res.ibm.com}}~\cite{ToledoGCFVLJAS2019}.
Here we send each sentence from one passage and the original query as the topic to the IBM Debater API for argument quality assessment. of Passages
The API then determines how good the quality of each argument with regards to the topic is with a \Bert-based regression classifier model trained on the IBM-ArgQ-6.3kArgs dataset. The model and therefore the API then returns a quality score ranging from 0 to 1 where a classified score of~0 means very poor argument quality and a score of~1 means very good argument quality~\cite{ToledoGCFVLJAS2019}.

As a second method to obtain the argument quality we use the T0++ language model~\cite{SanhWRBSACSLRDBXTSSKCNDCJWMSYPBWNRSSFFTBGBWR2021} again.
We ask the T0++ model to generate a text the following task: \query{<sentence>. How would you rate the readability and consistency in this sentence? very good, good, bad, very bad} where \query{<sentence>}~is one sentence of a passage.
This results in an output of either \query{very good}, \query{good}, \query{bad}, or \query{very bad} depending on how the pretrained T0++ model interprets the sentence's argument quality.
We then map this textual output labels to numeric values as per the mapping shown in Table~\ref{table-quality-mapping}.

\paragraph{Argument Stance Tagging}
\begin{table}
    \caption{Argument stance label mapping for textual labels returned by the T0++~language model~\cite{SanhWRBSACSLRDBXTSSKCNDCJWMSYPBWNRSSFFTBGBWR2021} for positive~(Pro) and negative~(Con) stance towards a single comparative object.}
    \label{table-stance-mapping}
    \begin{tabular}{llc}
        \toprule
        \multicolumn{2}{c}{\textbf{Text Label}} & \textbf{Value} \\
        Pro & Con & \\
        \midrule
        \query{yes} / \query{pro} & \query{yes} / \query{con} & \phantom{-}0 \\
        \query{yes} / \query{pro} & \query{no} & +1 \\
        \query{no} & \query{yes} / \query{con} & -1 \\
        \query{no} & \query{no} & \phantom{-}0 \\
        other & other & \phantom{-}0 \\
        \bottomrule
    \end{tabular}
\end{table}


Stance detection for each sentence uses the same conceptual approaches but with different inputs and outputs.
Since both the IBM Debater API~\cite{BarHaimBDSS2017} and our T0++ approach~\cite{SanhWRBSACSLRDBXTSSKCNDCJWMSYPBWNRSSFFTBGBWR2021} are only capable of calculating a single-target stance~(i.e., for one of the two comparative objects), we combine the two single-target stances into a multi-target stance by taking the difference between the stance towards the first comparative object and the stance towards the second comparative object.
We also experimented with different thresholds for the minimal difference between the single-target stances and found a threshold of~0.125 to work well when manually examining some classified examples.

For scoring the single-target argument stance for a sentence with the IBM Debater API, we again send the sentence and a claim built using one of the comparative objects to the IBM Debater API.
The classifier by \citet{BarHaimBDSS2017} computes an argument's likelihood of being pro, con, or neutral with respect to the claim~(i.e., the comparative object in our pipeline) by first classifying sentiments and then detecting contrasts in the topic and argument claim targets.
The API then returns a score from from~-1 to~+1 where -1~means the argument is against the comparative object and +1~means that the argument is in favor of the comparative object.
By classifying different claims for each object~(i.e., \query{<object>~is good} and \query{<object>~is the best}), we get an averaged single-target stance for each comparative object.

For the second method using the T0++~language model~\cite{SanhWRBSACSLRDBXTSSKCNDCJWMSYPBWNRSSFFTBGBWR2021} we first experiment with directly asking the model to generate~\query{pro}, \query{con}~or~\query{neutral} classification labels for a comparative object.
However, we found that the T0++ model would not be able to reliably distinguish between a positive and negative stance towards the comparative object.
Therefore we instead reformulate the task in two simple questions, one to determine whether the sentence has a positive stance towards the comparative object and one to determine whether it has a negative stance: \query{<sentence> Is this sentence pro <object>? yes or no} and \query{<sentence> Is this sentence against <object>? yes or no} where \query{<sentence>}~is one sentence of the passage and \query{<object>}~is one of the comparative objects.
This results in two answers~(\query{yes} or~\query{no}) for the positive and negative stance respectively. We combine the two textual answers as shown in Table~\ref{table-stance-mapping} and then combine the single-target stances into a multi-target stance as described above.

\subsection{Axiomatic Reranker}
\label{reranking}
\begin{table}
    \caption{Axioms used in our pipeline. Axioms without citations are our own work.}
    \label{table-axioms}
    \begin{tabular}{ll}
        \toprule
        \textbf{Name} & \textbf{Description} \\
        \midrule
        ArgUC~\cite{BondarenkoHVSPB2018} & Prefer more argumentative units. \\
        QTArg~\cite{BondarenkoHVSPB2018} & Prefer more query terms in argumentative units. \\
        QTPArg~\cite{BondarenkoHVSPB2018} & Prefer earlier query terms in argumentative units. \\
        CompArg & Prefer more comparative objects in argumentative units. \\
        CompPArg & Prefer earlier comparative objects in argumentative units. \\
        aSLDoc~\cite{BondarenkoFKHVS2019} & Prefer passages with 12--20 words per sentence. \\
        ArgQ & Prefer higher argument quality. \\ 
        \bottomrule
    \end{tabular}
\end{table}


Because we increase the recall of our retrieval system by expanding and reformulating queries~(c.f. Section~\ref{reformulation}), we re-rank top-10 result passages of our first-stage query likelihood retrieval~(c.f. Section~\ref{retrieval}).
We seek to improve our system's preciision by re-ranking with two different strategies that should rank passages higher that are more argumentative and higher quality, but also ensure a fair overview of the two conflicting comparative objects: \Ni we re-rank based on argumentative retrieval axioms and \Nii we re-rank based on the passages' stances towards the comparative objects.

\paragraph{Argumentative Axiomatic Re-ranking}

When only using frequncy-based ranking methods such as BM25 or query likelihood with Dirichlet smoothing, it is difficult to capture the inherent argumentativeness in passages.
This argumentativeness, however, is most important for finding relevant and useful opinions on comparative questions~\cite{BondarenkoFKSGBPBSWPH2022}.
Also, passages in our pipeline are already annotated with argumentative quality and stance.
Recent approaches for the TREC Common Core and Decision tracks exploit task-specific, argumentative retrieval axioms to ensure argumentativeness in their results~\cite{BondarenkoHVSPB2018,BondarenkoFKHVS2019}.
Axioms are constraints that define pairwise preferences between documents or passages.
Because of the promising development in the field of axiomatic information retrieval\footnote{The \iraxioms Python framework was only released shortly before our submission: \url{https://github.com/webis-de/ir_axioms/}}, we re-rank the top-10 passages with the \KwikSort algorithm~\cite{HagenVGS2016}.
For axiomatic re-ranking, we compute preferences for 7~argumentative axioms listed in Table~\ref{table-axioms}.
The axioms cover general argumentativeness~(ArgUC), argumentative relevance~(QTArg, QTPArg), comparative relevace~(CompArg, CompPArg), and rethorical and argumentative quality~(aSLDoc, ArgQ).
We then combine the axioms in a majority voting scheme, i.e., we only keep preferenes where at least~50\,\% of the 7~axioms agree, and fall back to the original ranking order if less than~50\,\% of all axioms agree.
Using the \iraxioms library, we then re-rank with the combined axiom.

\paragraph{Fairness Re-ranking}

We also implement a fairness reranker to produce rankings where the two conflicting stances~(pro first compared object and pro second compared object) are nearly equally prominent.
For balancing the argumentative stances, we experiment with two different re-ranking strategies: \Ni alternating stance and \Nii balanced top-\(k\) stance.
For the alternating stance strategy, we split the result set into three lists: First, with arguments in favor of the first comparative object, second, in favor of the second comparative object, and last, neutral arguments or arguments with no stance.
We then alternately select passages from the first two lists. If one or both lists are empty, we fall back to the neutral list.
The balanced top-\(k\) stance strategy is based on the original ranking.
Here we count the number of passages in favor of the first comparative object and the second comparative object in the top-\(k\) result set.
If the two numbers are imbalanced, i.e., their difference is greater than~1, we move the last passage from the majority within the top-\(k\) ranking behind the first minority passage after the top-\(k\) ranking.
This way, passages of the underrepresented stance advance the ranking until the ranking is balanced in the top-\(k\) positions.
In initial experiments, however, we find the alternating stance strategy to be more promising, because the balanced top-\(k\) stance strategy often lead to rankings containing mostly neutral passages.

\section{Evaluation}\label{evaluation}

\todo{Instead of submitting run files (i.e., data), in the Touché Lab on Argument Retrieval, participants must submit their retrieval systems as working software.}
We successfully deploy our software on the evaluation platform Tira\footnote{https://tira.io/} that \todo{shortly describe purpose of TIRA}.~\todocite
Our approach can be found under the name \todo{XXXX}.
The TIRA system automatically evaluates submitted approaches of all shared task participants and reports the nDCG@5 score.
The organizers use this site to ensure a reproducible leaderboard.
We will now have a closer look at the performance of our approach by looking at the top-5 retrieved passages for three chosen topics by two of the five submitted runs: 
(1) without query reformulation/expansion and axiomatic reranking and
(2) with query reformulation/expansion and axiomatic reranking.

\subsection{Topic 1}

The first query we will have a look at will be: \textit{Which is better, a laptop or a desktop?}
According to the narrative field provided by the organizers' passages that contain major similarities and dissimilarities of laptops and desktops are relevant.
Also relevant are passages which contain advantages and disadvantages of specific usage scenarios.

\paragraph{Run 1}

The first returned passage contains arguments for laptops and also for desktops while being subjective and non-biased.
Our second and fifth results are product descriptions of laptops and do not deliver arguments for or against laptops/desktops.
The third returned passage shows us results for laptops on a website where we can buy laptops.
Our fourth result gives us 5 arguments why laptops are better.
So manually we would  change the ranking as follows: 1-4-2-5-3

\paragraph{Run 2}

Our second run retrieves passages on its first three places which are about different laptops.
These websites give a detailed overview of the laptop they are talking about but no argument for or against laptops/desktops is given.
On our fourth place, we have a review for a network switch.
The fifth place is a selling website for electronic products.
Admittedly, none of the first five results are relevant or provide argumentative support to the user.
The passage which was ranked first in our first run is now on rank~34.
An explanation might be that using synonyms, in this case, lead to inaccurate search queries which did not capture the user's information need.

\subsection{Topic 2}

The next question we will have a look at is:
\textit{Is OpenGL better than Direct3D in terms of portability to different platforms?}
Relevant passages should contain information about the portability of OpenGL/Direct3D across different operation systems.
Not relevant are ads or tutorials for OpenGL/Direct3D.

\paragraph{Run 1}

The first result talks about 3D-rendering in the Opera browser and does not provide any arguments.
The second done is a gaming-related blog post again no arguments are given.
The third and fourth results are near-duplicates~\todocite and talk about a gaming developer who converted his game from OpenGL to Direct3D, including problems, tips, and anti-patterns while developing a 3D game.
In our last passage, we can get the information that Direct3D is now supported in Linux.
The only passages which are arguably relevant are the third and fourth.
All other of the first five passages are not relevant.

\paragraph{Run 2}

The returned results are the same passages returned by run 1 but in a different order.
The first two passages are the duplicate passages about game development and migrating from OpenGL to Direct3D.
Then we will get the passage which is about the support of Direct3D in Linux.
In fourth place we have the passage talking about the 3D-rendering in Opera and the last result is the gaming-related blog.
So the ranking has slightly improved since the most relevant passages are now in place 1 and 2 while other less relevant passages are now further behind in the ranking.

\subsection{Topic 3}

Our last topic will be:
\textit{Which technology performs better: Apple's or Google's?}
Relevant passages should compare both companies in terms of service and their products.
Relevant passages can focus on one company.
Not relevant passages are about genric information about the companies.

\paragraph{Run 1}

The first passage returned is a news site about technology and talks about various topics from this domain but no arguments are given.
The second passage talks about the new Google TV while AppleTV has been already released.
In the third place, there is a passage that talks about how Apple and Google both signed a privacy accord.
The fourth result is about Facebook buying Instagram.
The last result is about Google being caught violating users' privacy while Apple has been already caught violating users' privacy.
So we have some relevant passages here.
We would rank the passages as follows: 3-5-2-4-1.
So the top-5 ranking is not very good. 

\paragraph{Run 2}

The first passage is a news site about Apple products and provides some insight into these products.
The second passage provides information about selling numbers and stock prices.
So this site is not relevant according to the narrative provided by the organizers.
The next passage is a news article about how Google and Motorola have to hand over Android information to Apple.
So no arguments are provided in this passage.
The fourth result is about how Apple could only have success because of Google.
The last passage talks about five reasons iPhone vs Android is not Mac vs Windows.
So we can see that different passages have been retrieved for run 1 and run 2.
For run 2 we would rank the passages as follows: 1-4-5-3-2.

\todo{Little summary of manual evaluations.}
\section{Conclusion}

We propose a flexible approach to tackle the task of answering comparative questions and detecting the stance of the returned passages with respect to the comparative objects.
Our approach combines query reformulation and expansion techniques with axiomatic reranking exploiting argumentative structure, quality, and stance.
To determine argument quality and stance we showcase two state-of-the-art approches using the IBM Debater API and the T0++ language model.
Our work builds on approaches from earlier years of the Touché shared task by using different techniques to expand the original queries and also incorporates the contextual information provided in topic descriptions and narratives.
Furthermore, we propose two simple strategies to balance the exposure of conflicting argument stances on top-\(k\) ranks.

Our experiments from Section~\ref{evaluation} show that query expansion and reformulation using additional information from descriptions and narratives can provide better search results for some topics, but can also decrease effectiveness for others.
A subsequent axiomatic re-ranking step can improve the system's precision but we do not include a systematic evaluation of the re-ranker.

Since we solely rely on external APIs for stance classification that only provide a single-target stance, computing the multi-target stance from a single-target stance proved to be challenging.
Second, our current approach is unable to distinguish neutral arguments from arguments with no stance at all.
We argue that fine-tuning a neural network like \Bert on the stance dataset provided by \citeauthor{BondarenkoFKSGBPBSWPH2022} could improve classification performance by directly predicting the multi-target stance.

Another interesting question that arose during our research is if it would be possible to mainly use large language models to develop an argumentative information retrieval pipeline.
Given the recent controversy of using large language models in information retrieval, we look forward to evaluate this question in more detail with relevance judgements for the Touché 2022 shared task.

Our approach showcases several possibile solutions to the diverse and challenging problems in answering comparative questions and highlights limitations and caveats.


\bibliography{../literature/literature.bib}

\end{document}

%%
%% End of file
